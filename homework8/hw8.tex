\documentclass[]{article}

\usepackage{caption}
\usepackage{graphicx, subfig}
\usepackage{listings}
\usepackage[namelimits]{amsmath} 
\usepackage{fontspec}
\usepackage{amsmath}
\usepackage{amssymb}                      
\usepackage{mathrsfs}  
\usepackage{amsfonts}   
\setmainfont[Mapping=tex-text]{KaiTi}
\usepackage{fullpage}
\usepackage{amsthm}
\usepackage{fancyhdr}
\usepackage{algorithm}
\usepackage{algorithmic}
\usepackage{bm}
\usepackage{ctex}
\usepackage{txfonts}
\usepackage{tikz}
\usetikzlibrary{shapes.geometric, arrows}


%opening
\title{统计机器学习 课后作业}
\author{陈劭涵 17300180049}



\newcommand{\tm}{\fontspec{Times New Roman}}


\begin{document}
	
\maketitle


\section{问题 1}
\begin{flushleft}
解:
\end{flushleft}
将五个样本按行从左到右的顺序分别命名为1,2,3,4,5\\\\
初始化有五个类别,命名为A-E,分别为:\\\\
$A:1\\
B:2\\
C:3\\
D:4\\
E:5\\
$\\
1层聚类后,$d_{CE}=1$最小,故将$C$和$E$聚为一个类,现在的类别为:\\\\
$A:1\\
B:2\\
D:4\\
F:3,5\\
$\\
2层聚类后,$d_{AF}=2$最小,故将$A$归到$F$中形成新的类,现在的类别为:\\\\
$B:2\\
D:4\\
G:1,3,5\\
$\\
3层聚类后,$d_{BD}=4$最小,故将$B$归到$D$中形成新的类,现在的类别为:\\\\
$G:1,2,3,5\\
H:2,4\\
$\\
4层聚类后,将$G$和$H$合成一个类:\\\\
$I:1,2,3,4,5\\
$\\
此时类的个数为1,即所有样本被归类到一个类中,故层次聚类结束\\\\

\section{问题 2}
\begin{flushleft}
解:
\end{flushleft}
1.首先选择两个样本点作为类的中心,按题意我们选取$m_1^{(0)}=x_4=(5,0)^T$,$m_2^{(0)}=x_5=(5,2)^T$\\\\
2.然后以$m_1^{(0)}$和$m_2^{(0)}$作为类$G_1^{(0)}$与类$G_2^{(0)}$的中心,计算$x_1$, $x_2$, $x_3$与$m_1^{(0)}$和$m_2^{(0)}$的欧氏距离平方\\\\
对$x_1=(0,2)^T$, $d(x_1,m_1^{(0)})=29$, $d(x_1,m_2^{(0)})=25$,将$x_1$分到类$G_2^{(0)}$\\\\
对$x_2=(0,0)^T$, $d(x_2,m_1^{(0)})=25$, $d(x_2,m_2^{(0)})=29$,将$x_2$分到类$G_1^{(0)}$\\\\
对$x_3=(1,0)^T$, $d(x_3,m_1^{(0)})=16$, $d(x_3,m_2^{(0)})=20$,将$x_3$分到类$G_1^{(0)}$\\\\
3.得到新的类$G_1^{(1)}=\{x_2,x_3,x_4\}$, $G_2^{(1)}=\{x_1,x_5\}$,计算类的中心$m_1^{(1)}$, $m_2^{(1)}$\\\\
$m_1^{(1)}=\{2,0\}^T$, $m_2^{(1)}=\{2.5,2.0\}^T$\\\\
4.重复步骤2和3,然后将$x_1$,$x_5$分到类$G_2^{(1)}$, 将$x_2$, $x_3$, $x_4$分到类$G_1^{(1)}$,得到新的类:\\\\
$G_1^{(2)}=\{x_2,x_3,x_4\}$, $G_2^{(2)}=\{x_1,x_5\}$\\\\
发现新的类没有发生改变,故聚类停止,得到聚类结果:\\\\
$G_1^{(*)}=\{x_2,x_3,x_4\}$, $G_2^{(*)}=\{x_1,x_5\}$
\end{document}
